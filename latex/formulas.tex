
\documentclass[11pt,a4paper]{article}

% Essential packages
\usepackage[utf8]{inputenc}
\usepackage[T1]{fontenc}
\usepackage{lmodern}
\usepackage{amsmath,amssymb,amsfonts}
\usepackage{microtype}
\usepackage[margin=1in]{geometry}
\usepackage{xcolor}
\usepackage{hyperref}

% Minimal styling
\hypersetup{
    colorlinks=true,
    linkcolor=blue,
    filecolor=blue,
    citecolor=blue,
    urlcolor=blue,
}

% Custom title formatting
\usepackage{titlesec}
\titleformat*{\section}{\Large\bfseries}
\titleformat*{\subsection}{\large\bfseries}
\titleformat*{\subsubsection}{\normalsize\bfseries}

% Document info
\title{\vspace{-2em}Document Title}
\author{Author Name}
\date{\today}

\begin{document}

\maketitle

\section{Introduction}
This is a clean, minimalist template designed for professional documents. It provides good readability while maintaining a modern appearance.

\section{ISPC Formula}
The Intersite Phase Clustering (ISPC) is defined as:

\begin{equation}
\text{ISPC of } f = \left| \frac{1}{n} \sum_{t=1}^{n} e^{i(\phi_{xt} - \phi_{yt})} \right|
\end{equation}

Where:
\begin{itemize}
    \item $n$ is the number of time points
    \item $\phi_{xt}$ is the phase angle of signal $x$ at time $t$
    \item $\phi_{yt}$ is the phase angle of signal $y$ at time $t$
\end{itemize}

It is the average of phase angle differences between signals over time. \\

Implementation in MATLAB:

\begin{verbatim}
ISPC = abs(mean(exp(1i * (angles1 - angles2))));
\end{verbatim}


ISPC can be computed using a over time using a sliding window approach in a similar manner to the way FFT is computed in short-time FFT. The selection of the time segment length depends on the frequency and task. Longer segments gives better signal-to-noise ration but has worse time resolution. \\


\end{document}

\text{ISPC of } f = \left| \frac{1}{n} \sum_{t=1}^{n} e^{i(\phi_{xt} - \phi_{yt})} \right
